\documentclass[12pt, letterpaper]{article}
\usepackage[utf8]{inputenc}
\usepackage{graphicx}
\usepackage{listings}
\usepackage{verbatim} 
\title{Rapport de Projet : ImagesTaches}
\author{Chaolei CAI \& Nawfoel Arjoune}

\begin{document}


\begin{titlepage}
\maketitle
\end{titlepage}

\tableofcontents
\section{Introduction}
Ce document constitue notre rapport sur le projet de fin de semestre L2 pour le cours d'impérative 2 de M.Bourdin. \\
Le projet est consultable depuis ce lien github: https://github.com/bk211/ImagesTaches\\


\section{Dependences}
Ce projet nécessite le support de l'API GL4Dummies de M.Belhadj, 
par extension, vous devez avoir sur votre machine les dépendances de GL4Dummies 
(SDL2 et OpenGL pour ne citer que les plus importants).

\section{Compilation du projet}
Un makefile est présent dans le répertoire de racine, pour compiler le programme, une simple
commande "make" suffit à obtenir l'exécutable "exec".\\
Enfin, lancer le programme exec pour faire appraître la fênetre d'affichage.\\
Par défaut l'image affiché est l'image avant le traitement, 
pour afficher le l'image après application du traitement, pressez la touche "r" de votre clavier.\\
Pour terminer le programme, vous pouvez pressez la touche "ECHAP", "q" ou la croix situé dans le coin supérieur droite.
\section{Objectives et consigne du Projet}
Il faut sur, une image, trouver automatiquement toutes les "taches de couleur".\\
L'image est donnée comme un grand tableau avec un octet par couleur par "pixel" (trois couleurs, R, G et B).\\
C'est donc une vaste matrice de (XMAX x YMAX x 3) octets. Trouver les taches consiste à trouver et numéroter les zones connexes ayant la même couleur.\\
Une bonne méthode consisterait à commencer par trouver tous les pixels connexes ayant une certaine couleur, ils forment une tache.\\
Puis à réitérer ce processus pour tous les pixels. On a alors toutes les taches, si elles ont été listées, il est maintenant facile de leur appliquer un traitement.\\
Un traitement pourrait être de délimiter en noir les taches de couleur. Enfin, une tache peut être une zone dont les pixels ont à peu près la même couleur.\\
On définira cette approximation et on fera la délimitation de ces zones comme précédemment. \\
Le rapport contiendra des explications sur les procédures mises en oeuvre, le mode d'emploi, le listing du programme et des traces d'utilisation.\\
Le programme devra tourner sur au moins une des machines en libre-service du Bocal et sera écrit en C, avec, si nécessaire, 
utilisation des librairies OpenGL, Glu, Glut et GL4D, seules librairies non standard autorisées.

\section{Structures de données de base}
\begin{lstlisting}
    typedef unsigned char color;

    typedef struct pixel pixel;
    struct pixel{
        color R;
        color G;
        color B;
    };

    typedef struct image image;
    struct image{
        color * tab;
        int w;
        int h;
    };
        
\end{lstlisting}
Tout d'abord, "color" est un alias que j'ai donné au type unsigned char (il occupe précisément 1 octet dans la mémoire), il a été mise en place pour 
quantifier la couleur dans le système RGB, en effet, dans ce dernier, les valeurs que peuvent prendre
une couleur est comprise dans l'intervalle [0;255].\\
Par extension, on arrive à notre structure pixel, qui est constitué de 3 composants "color" RBG. \\
Enfin, une image seras pour nous une structure composé de 3 élements:\\
Le premier, "tab" est le tableau de couleur qu'il faudra alloué dynamiquement selon le besoin.\\
"w" et "h" sont les variables qui indique les dimensions de l'image.\\
J'ai choisi de prendre un tableau à simple entrée comme nous avons déja les dimensions de l'image, 
la manipulation des indices n'est pas fondamentalement plus dur d'un tableau à 2 entrée, ce n'est qu'une 
histoire de conversion entre les coordonnées x et y.\\
A vrai dire, dans les première versions, il s'agit plutôt d'un pointeur vers une structure "pixel" que nous avons utilisée.
Mais comme dans la consigne, il est dit qu'il faut utiliser une matrice de (XMAX x YMAX x 3) octets, nous sommes revenue vers le pointeur vers "color".


\end{document}
